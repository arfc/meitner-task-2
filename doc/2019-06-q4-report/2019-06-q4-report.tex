\documentclass[12pt]{article} %openany
\usepackage{fullpage}
\usepackage[affil-it]{authblk}
\usepackage[english]{babel}
\usepackage{graphicx}
\usepackage{rotating}
%\usepackage{bibtex}
%\usepackage[utf8]{inputenc}
%\usepackage[english]{babel}
\usepackage{adjustbox}
\usepackage{courier}
\usepackage{verbatim}
\usepackage{url}
\usepackage[hidelinks]{hyperref}
\usepackage{float}
\usepackage{array}
\usepackage{breakcites}
\usepackage{gensymb}
%\usepackage[backend=biber]{biblatex}
\usepackage{multirow} 
\usepackage{placeins}
\usepackage{booktabs}
\usepackage{tabularx}
\newcolumntype{b}{>{\hsize=1.0\hsize}X}
\newcolumntype{s}{>{\hsize=.5\hsize}X}
\newcolumntype{m}{>{\hsize=.75\hsize}X}
\newcolumntype{x}{>{\hsize=.25\hsize}X}
\newcolumntype{a}{>{\hsize=.11\hsize}X}
\graphicspath{{figures/}}
\usepackage{listings}
\usepackage{appendix}
\usepackage{cite}
\usepackage{blindtext}
\usepackage[utf8]{inputenc} % Required for inputting international characters
\usepackage[T1]{fontenc} % Output font encoding for international characters
\usepackage{mathpazo} % Palatino font
\usepackage{graphicx} % For the logo
\usepackage[acronym,toc]{glossaries}
\include{acros}
\usepackage{amssymb}
\usepackage{pifont}
\usepackage{ragged2e}
\usepackage{xcolor}
\newcommand{\greencheck}{{\color{green}\checkmark}}
\newcommand{\xmark}{{\color{red}\ding{55}}}

%\bibliographystyle{numeric}

\begin{document}

%----------------------------------------------------------------------------------------
%    TITLE PAGE
%----------------------------------------------------------------------------------------

\begin{titlepage} % Suppresses displaying the page number on the title page and the subsequent page counts as page 1
    \newcommand{\HRule}{\rule{\linewidth}{0.5mm}} % Defines a new command for horizontal lines, change thickness here
    
    \center % Centre everything on the page

    %------------------------------------------------
    %    Title
    %------------------------------------------------
    
    \HRule\\[0.2cm]
    
     \begin{minipage}{0.4\textwidth}
        \includegraphics[width=\textwidth]{arfc-logo}
        \end{minipage}%
        \begin{minipage}{0.6\textwidth}
        {\begin{flushright}\huge\bfseries Milestone 2.1  Enabling Load Following Capability in the Transatomic Power MSR \end{flushright}}
        {\begin{flushright}\large\textit{Modeling-Enhanced Innovations Trailblazing Nuclear Energy Reinvigoration (MEITNER) DE-FOA-0001798 \\ Task 2 Milestone Report}\end{flushright}}

        \end{minipage}

    \vspace{0.2cm}
    \HRule
    \vspace{0.5cm}
    
    %------------------------------------------------
    %    Author(s)
    %------------------------------------------------
    
    \begin{minipage}{0.4\textwidth}
        \begin{flushleft}
            \large
            \textit{Author}\\
            Andrei \textsc{Rykhlevskii}\\
        \end{flushleft}
    \end{minipage}
    ~
    \begin{minipage}{0.4\textwidth}
        \begin{flushright}
            \large
            \textit{Principal Investigator}\\
            Kathryn D. \textsc{Huff} % Supervisor's name
        \end{flushright}
    \end{minipage}
    
    % If you don't want a supervisor, uncomment the two lines below and comment the code above
    %{\large\textit{Author}}\\
    %John \textsc{Smith} % Your name

    %------------------------------------------------
    %    Report Number
    %------------------------------------------------
    \vspace{1cm}
    \textsc{\LARGE\bfseries UIUC-ARFC-2019-012} % Replace YYYY with the year, NN with report index
    \vspace{0.5cm}
    
    %------------------------------------------------
    %    Date
    %------------------------------------------------
    
    \vspace{0.5cm} % Position the date further down the remaining page
    {\large\today} % Date, change the \today to a set date if you want to be precise
    \vspace{0.5cm}

%------------------------------------------------
    %    Headings
    %------------------------------------------------
    
    \textsc{\LARGE Advanced Reactors and Fuel Cycles}\\[0.25cm] % Research Group
    
    \textsc{\large Dept. of Nuclear, Plasma, \& Radiological Engineering}\\% Department
    
    \textsc{\large University of Illinois at Urbana-Champaign}\\ % University

    %------------------------------------------------
    %    Logo
    %------------------------------------------------

    \vspace{0.5cm}
    \includegraphics[width=0.8\textwidth]{illinois}\\[1cm] % Include a department/university logo - this will require the graphicx package

    %------------------------------------------------
    %   Funding 
    %------------------------------------------------
    % For this section, either use \vfill to fill the space 
    % or insert funding acknowledgement
    \textit{This research is being performed using funding received from the 
    Department of Energy ARPA-E MEITNER Program (award DE-AR0000983).}

\end{titlepage}

\section{Introduction}
We initiated the Fuel Cycle Simulation task (Task 2) in August 2018 to more 
realistically model the online reprocessing system of the \gls{TAP} 
\gls{MSR}. A Python toolkit, SaltProc v1 \cite{rykhlevskii_modeling_2019,
rykhlevskii_advanced_2018, rykhlevskii_arfc/saltproc_2018}, was developed to 
represent the simplified online fuel salt processing of a \gls{MSBR}.
More recently, an advanced SaltProc version (SaltProc v2.0+) was developed 
to simulate the complex salt reprocessing system of the \gls{TAP}, 
incorporating user-parametrized components into the fuel salt processing 
design. This report summarizes the progress we have made towards milestone 
\textbf{M2.1: Demonstration SaltProc}, the challenges we currently face, and 
the steps towards ultimate Task 2 objectives.

\section{Milestone objectives}
ARPA-E Award No. DE-AR0000983 with the Board of Trustees of the University of Illinois Attachment 3 (Technical Milestones and Deliverables) section D (Description of 
technical tasks, milestones, and deliverables) formulated M2.1 goal as follows: \\
``Initial demonstration of fuel cycle simulation package working together with 
Monte Carlo to complete full core TAP reactor depletion calculation. SaltProc 
will use separations efficiencies and dynamics based on work in Task 1 and will 
be coupled with Serpent 2 where Monte Carlo results will be done to <10\% 
relative error accuracy.'' \\ Herein we demonstrated the following 
capabilities of SaltProc v2.0+:
\begin{enumerate}
	\item Read a user-defined Serpent 2 input template file with the model geometry, 
	material composition, total heating power, and boundary conditions.
	\item Read a user-defined \emph{.json} input file with the parameters and 
	structure of the fuel salt reprocessing system.
	\item Run Serpent 2 in parallel mode to perform a depletion calculation.
	\item Read the depleted fuel composition file and store it in the HDF5 
	database \cite{the_hdf_group_hierarchical_1997}.
	\item Remove poisons from the fuel's isotopic composition by passing 
	information	through the user-parametrized components of the fuel salt 
	processing system. For demonstration proposes, SaltProc v2.0+ used 
	user-defined constant 	separation efficiencies but can handle variable 
	efficiencies defined in Task 1.
	\item Replace fuel salt mass lost in the primary loop due to poisons 
	extraction by adding fresh salt with a user-defined isotopic composition 
	(e.g., \gls{LEU} 5\% and 19.79\%, for this work).
	\item Record the fuel salt composition after salt reprocessing; waste 
	streams from each component of the reprocessing system; and other major 
	core parameters, such as multiplication factor, burnup, total fissile 
	mass, effective delayed neutron fraction, and breeding ratio.	
\end{enumerate}

\section{The TRANSATOMIC POWER Molten Salt Reactor concept}
The \gls{TAP} concept is a 1250 MW$_{th}$ \gls{MSR} with a LiF-based uranium 
fuel salt \cite{transatomic_power_corporation_technical_2016}. This concept 
uses configurable zirconium hydride (ZrH$_{1.66}$) rods as the moderator while 
most \gls{MSR} designs usually propose high-density reactor graphite. 
Zirconium hydride can achieve the same degree of thermalization as graphite 
with a much smaller volume. Compared to graphite, which shrinks and swells 
over time under irradiation, the cladded zirconium hydride has a much 
longer lifespan in extreme operational conditions - high temperature, large 
neutron flux, chemically aggressive salt. Finally, zirconium hydride is a 
nonporous material that absorbs much fewer neutron poisons (e.g., krypton, 
xenon) compared to high-density reactor graphite 
\cite{transatomic_power_corporation_technical_2016, 
transatomic_power_corporation_neutronics_2016, betzler_two-dimensional_2016}.

\subsection{TAP design description}
The \gls{TAP} design (figure~\ref{fig:tap-main-view}) is very similar to the  
original \gls{MSRE} design developed by \gls{ORNL} 
\cite{haubenreich_experience_1970} but with two major innovations: 
the fuel salt composition and the moderator. The LiF-BeF$_2$-ZrF$_4$-UF$_4$ 
salt used in \gls{MSRE} has been substituted with a LiF-UF$_4$ salt allowing  
the uranium concentration within the fuel salt to be increased from 0.9 to 
27.5\% while maintaining a relatively low melting point (490$^{\circ}$C 
compared with 434$^{\circ}$C for the original \gls{MSRE}'s salt) 
\cite{betzler_two-dimensional_2016}. The graphite has a very high 
thermal scattering cross section which would make it an excellent moderator 
but it has a few major drawbacks. First, due to the low lethargy gain per 
collision, the core requires a large volume of graphite to reach criticality,  
leading to a larger core and obstructing the core power density. Second, even 
special reactor-grade graphite has relatively high porosity, meaning, it holds
gaseous \glspl{FP} (e.g., tritium, xenon) in its pores. Third, the reactor 
graphite lifespan in a commercial reactor is only 10 years 
\cite{robertson_conceptual_1971}. To resolve these issues, the \gls{TAP} 
concept uses an alternative moderator, zirconium hydride, allowing for 
a more compact core and a significant increase in power density. These two 
innovative design choices, together with a configurable moderator 
(the moderator-to-fuel ratio can be changed during regular maintenance 
shutdown), facilitate the commercial deployment of this conceptual design 
viable in the commercially available 5\% \gls{LEU} fuel cycle. 
\begin{figure}[htp!] % replace 't' with 'b' to 
  		\hspace{+1.6in}
		  \includegraphics[width=0.65\textwidth]{tap_front_view.png}
  \caption{Schematic view of the \gls{TAP} \gls{MSR}showing the movable 
  moderator rod bundles and the shutdown rod (figure reproduced from 
  Transatomic Power White Paper 
  \cite{transatomic_power_corporation_technical_2016}).}
  \label{fig:tap-main-view}
\end{figure}

The primary loop of the \gls{TAP} \gls{MSR} consists of the reactor core 
volume moderated by the silicon carbide (SiC) cladded zirconium hydride rods, 
pumps, and primary heat exchanger. The pumps circulate the LiF-(Act)F$_4$ fuel 
salt through the primary loop. The pumps, vessels, tanks, and piping are made 
of a corrosion resistant nickel-based alloy (similar to Hastelloy-N\footnote{ 
Hastelloy-N is very common in reactors now but have been studied and developed 
at \gls{ORNL} in a program that started in 1950s.}) in various molten salt 
environments. Inside the reactor vessel, near to the zirconium hydride 
moderator rods, the fuel salt is in a critical configuration and generates 
heat. Table~\ref{tab:tap_tab} contains details of the \gls{TAP} system design  
taken from the technical white paper 
\cite{transatomic_power_corporation_technical_2016}, the neutronics overview
 \cite{transatomic_power_corporation_neutronics_2016}, and the \gls{ORNL} 
 analysis of the \gls{TAP} design \cite{betzler_two-dimensional_2016, 
 betzler_assessment_2017}. 
%%%%%%%%%%%%%%%%%%%%%%%%%%%%%%%%%%%%%%%%
\begin{table}[h!]
        \caption{Summary of principal data for the \gls{TAP} \gls{MSR} 
        (reproduced from \cite{transatomic_power_corporation_technical_2016, betzler_assessment_2017}). }
        \begin{tabularx}{\textwidth}{ s  s}
        \hline
         Thermal power				           		& 1250 MW$_{th}  $       \\ 
         Electric power		                		& 520 MW$_e  $ 			 \\ 
         Gross thermal efficiency        			& 44\%     				 \\  
         Outlet temperature							& 620$^{\circ}$C         \\ 
		 Fuel salt components                   & LiF-UF$_4$				 \\  
 		 Fuel salt composition                  & 72.5-27.5 mole\%			 \\  
         Uranium enrichment                     & 5\% $^{235}$U          	 \\
         Moderator                              & Zirconium Hydride (ZrH$_{1.66}$) rods (with silicon carbide cladding) \\
	     Neutron spectrum						& 
	     Thermal/Epithermal                 \\
         \hline
        \end{tabularx}
        \label{tab:tap_tab}
\end{table}
%%%%%%%%%%%%%%%%%%%%%%%%%%%%%%%%%%%%%%%%%%%%%%%%
\subsection{TAP core design}
In the \gls{TAP} core (Figure~\ref{fig:tap-core-view}), lattices of SiC clad 
moderator rods form the moderator assemblies around which the fuel salt flows  
(Figure~\ref{fig:tap-main-view}). The \gls{TAP} reactor pressure vessel is a 
cylinder with an inner radius of 150 cm, a height of 350 cm, and a wall 
thickness of 5 cm. The moderator-to-fuel ratio, or salt volume fraction 
(SVF), in the core can be varied during operation to shift the spectrum from 
intermediate to thermal energies to maximize fuel burnup. Intermediate 
energies are used at \gls{BOL} and are shifted to thermal at \gls{EOL}. During 
operation the SVF can be varied by inserting fixed-sized moderator rods from 
the bottom of the reactor vessel, similarly to moving the control rods in 
a \gls{BWR}, as shown in Figure~\ref{fig:tap-main-view}. For the \gls{TAP} 
reactor, \gls{EOL} occurs when the maximum number of moderator rods are 
inserted into the core and further injection of fresh fuel salt does not 
change a criticality. Unmoderated salt flowing in the annulus between the core 
and the vessel wall provides potential reduction of fast neutron flux at the 
vessel structural material  
\cite{transatomic_power_corporation_neutronics_2016}.
\begin{figure}[t] % replace 't' with 'b' to 
		  \includegraphics[width=\textwidth]{tap_core_ornl.png}
	  	\vspace{-0.35in}
  \caption{The \gls{TAP} \gls{MSR} schematic core view showing moderator rods 
  (figure reproduced from ORNL/TM-2017/475  
\cite{betzler_assessment_2017}).}
  \label{fig:tap-core-view}
\end{figure}

\subsection{TAP reprocessing system structure and simulation approach}
The \gls{TAP} nuclear island contains a \gls{FP} removal system. Gaseous 
\glspl{FP} are continuously removed using an off-gas system while liquid 
and solid \glspl{FP} are extracted via a chemical processing system. A small 
quantity of fresh fuel salt is regularly added to the primary loop as  
byproducts are gradually removed. This process maintains a constant fuel salt 
mass and keeps the reactor critical. In contrast with the \gls{MSBR} 
reprocessing system, the \gls{TAP} does not require a protactinium separation 
and isolation system because it operates in a single-stage uranium-based fuel 
cycle. The authors of the \gls{TAP} concept detailed three distinct fission 
product removal methods \cite{transatomic_power_corporation_neutronics_2016}:
\paragraph{Off-Gas System:} Gaseous fission products such as krypton 
and xenon are removed, compressed, and stored temporarily until they have 
decayed to background radiation levels. Trace amounts of tritium are also 
removed and bottled in a liquid form via the same process. The off-gas system 
also removes a small fraction of the noble metals.
\paragraph{Metal Plate-Out/Filtration:} Removes solid noble and semi-noble 
metal fission products as they plate out onto a nickel mesh filter located in 
a side stream of the primary loop.
\paragraph{Liquid Metal Extraction:} Lanthanides and other non-noble 
metals stay dissolved in the fuel salt. These elements generally have a lower 
capture cross section and thus absorb fewer neutrons than $^{135}$Xe but their 
extraction is essential to ensuring normal operation. In the \gls{TAP} 
reactor, lanthanide removal is accomplished via a liquid-metal/molten salt 
extraction process similar to that developed for \gls{MSBR} by \gls{ORNL} 
\cite{robertson_conceptual_1971}. The process converts the dissolved 
lanthanides into a well-understood oxide waste form, similar to that for  
\gls{LWR} \gls{SNF}. This oxide waste comes out of the \gls{TAP} reprocessing 
plant in ceramic granules, which can be sintered into another convenient form 
for storage.

Figure~\ref{fig:tap-reproc} shows a principal design of the \gls{TAP} primary 
loop including an off-gas system, nickel mesh filter, and lanthanide chemical 
extraction facility. Similarly to \gls{MSBR}, the off-gas system is based on a 
simple process of helium sparging through the fuel salt with consequent gas 
bubbles removed before returning the fuel salt back to the core. One very 
notable difference is the \gls{MSBR} gas separation system helium injection 
and subsequent transport of the voids run throughout the primary loop, 
including the core, for at least 10 full loops 
\cite{robertson_conceptual_1971}. This system presents a significant concern 
to the safety and stability of operation due to the increase of void fraction 
in the fuel salt when it enters back to the core, causing unpredictable 
changes in reactivity. This drawback can be overcome by using an effective gas 
separator to strip helium/xenon bubbles before returning the salt back to a 
primary loop (Figure~\ref{fig:tap-reproc}, blue block). 
\begin{figure}[htp!] % replace 't' with 'b' to 
  \centering
		  \includegraphics[width=\textwidth]{tap_primary_loop.png}
  \caption{Simplified \gls{TAP} primary loop design including off-gas system 
  (blue), nickel filter (orange) and liquid metal extraction system (green) 
  (reproduced from \cite{transatomic_power_transatomic_2019}).}
  \label{fig:tap-reproc}
\end{figure}

Solid noble and semi-noble metal fission products tend to plate out onto the  
metal surfaces including piping, heat exchanger tubes, reactor vessel inner 
surface, etc. Previous research by \gls{ORNL} \cite{robertson_conceptual_1971} 
concluded that about 50\% of noble and semi-noble metals would plate out 
inside \gls{MSBR} systems without any special treatment. To improve the 
extraction efficiency of these fission products, the \gls{TAP} concept employs 
a nickel mesh filter located in a bypass stream in the primary loop 
(Figure~\ref{fig:tap-reproc}, orange block). The main idea of this filter is 
to create a maze with a large metal (nickel) surface area. The fuel salt flows 
throughout the filter and the noble metals plate-out on the filter internal 
surface. 

This Liquid Metal Extraction process for the \gls{TAP} concept has been 
adopted from the \gls{MSBR}. The \gls{MSRE} demonstrated a liquid-liquid 
extraction process for removing rare earths and lanthanides from the fuel salt 
and estimated its efficiency.

The \gls{TAP} project reported a detailed list of elements for removal and 
removal efficiencies (Table~\ref{tab:reprocessing_list}). We used data from 
\gls{TAP} neutronics 
whitepaper\cite{transatomic_power_corporation_neutronics_2016} for the 
SaltProc v2.0+  demonstration case without any modifications.
%%%%%%%%%%%%%%%%%%%%%%%%%%%%%%%%%%%%%%%%
\begin{table}[ht!]
        \centering
        \caption{The effective cycle times for fission products removal from 
        the \gls{TAP} \gls{MSR} (reproduced from 
        \cite{betzler_implementation_2017} and 
        \cite{transatomic_power_corporation_neutronics_2016}).}
        \begin{tabular}{p{0.2\textwidth} p{0.42\textwidth} p{0.12\textwidth} p{0.16\textwidth}}
        \hline 
        %\begin{tabularx}{\linewidth}{l X} \toprule 
        Processing group & \qquad\qquad\qquad Nuclides & Removal Rate (s$^{-1}$) & Cycle time (at full power) \\ [5pt] \hline 
 \multicolumn{3}{c}{\textit{Elements removed in \gls{MSBR} concept and adopted for the \gls{TAP}} \cite{robertson_conceptual_1971}} \\
        Volatile gases & Xe, Kr								  & 5.00E-2 & 20 sec \\ [5pt]
        Noble metals & Se, Nb, Mo, Tc, Ru, Rh, Pd, Ag, Sb, Te & 5.00E-2 & 20 sec \\ [5pt]
        Seminoble metals & Zr, Cd, In, Sn	  				  & 5.79E-8 & 200 days \\ [5pt]
        Volatile fluorides & Br, I 							  & 1.93E-7 & 60 days \\ [5pt]
        Rare earths & Y, La, Ce, Pr, Nd, Pm, Sm, Gd           & 2.31E-7 & 50 days \\ [5pt]
        \qquad & Eu & 2.32E-8 & 500 days \\ [5pt]
        Discard & Rb, Sr, Cs, Ba & 3.37E-9 & 3435 days \\ [5pt] 
        \hline
 
 \multicolumn{3}{c}{\textit{Additional elements removed} \cite{transatomic_power_corporation_neutronics_2016, betzler_implementation_2017}  } \\
        Volatile gases & H								  	& 5.00E-2 & 20 sec    \\ [5pt]
        Noble metals & Ti, V, Cr, Cu						& 3.37E-9 & 3435 days \\ [5pt]
        Seminoble metals & Mn, Fe, Co, Ni, Zn, Ga, Ge, As   & 3.37E-9 & 3435 days \\ [5pt]
        Rare earths & Sc									& 3.37E-9 & 3435 days \\ [5pt]
        Discard & Ca										& 3.37E-9 & 3435 days \\ [5pt] 
        \hline
        \end{tabular}
        \label{tab:reprocessing_list}
          \vspace{-0.9em}
\end{table}

We simulated \gls{TAP} \gls{MSR} depletion in SaltProc v2.0+ using  
reprocessing cycle times from Table~\ref{tab:reprocessing_list}. The online 
reprocessing system design details, and a full-core reactor Serpent model 
(section~\ref{sec:tap_model}) to capture the dynamics of fuel composition 
evolution during reactor operation.

\section{The SaltProc modeling and simulation code} \label{sec:tool}
The first version of the SaltProc Python tool was developed in 2018 as a part 
of M.S. thesis to calculate \gls{MSR} fuel composition evolution taking into 
account an online reprocessing system \cite{rykhlevskii_advanced_2018,
rykhlevskii_arfc/saltproc_2018}. The tool was designed to expand depletion 
capabilities of Serpent 2 for modeling liquid-fueled \gls{MSR} with an online 
fuel reprocessing system. SaltProc v1 uses HDF5  
\cite{the_hdf_group_hierarchical_1997} to store data and uses the PyNE Nuclear 
Engineering Toolkit \cite{scopatz_pyne_2012} to parse Serpent 2 output. 
SaltProc v1 is an open-source Python package that uses a batch-wise approach 
to simulate continuous feeds and removals in \glspl{MSR}. 

SaltProc v1 only allows 100\% separation efficiency for either specific 
elements or groups of elements (e.g., Processing Groups as described in 
Table~\ref{tab:reprocessing_list}) at the end of the specific cycle time. This 
simplification neglects the reality that the salt spends an appreciable amount 
of time out of the core, in the primary loop pipes and heat exchanger. This 
approach works well for fast-removing elements (gases, noble metals) which 
should be removed after each depletion step. Unfortunately, for the elements 
with longer cycle times (i.e. rare earths which should be removed every 50 
days) this simplified approach leads to oscillatory behavior of all major 
parameters \cite{rykhlevskii_modeling_2019}. 

The capabilities of the SaltProc, paired with the Monte Carlo software 
Serpent 2, were demonstrated using the full-core MSBR design for a simplified  
case using ideal removal efficiency (100\% of mass for target elements 
removed) \cite{rykhlevskii_modeling_2019}. The preliminary version of the 
SaltProc architecture and principal structure were not designed for flexible 
implementation of sophisticated online reprocessing systems, including 
realistic physics/chemistry-based extraction efficiencies. 

We completely re-factored SaltProc v1 using \gls{OOP} to create a generic, 
comprehensive tool to realistically model any \gls{MSR} reprocessing plant 
while taking into account non-ideal or variable extraction efficiencies and 
mass balance between the core and processing plant.

\subsection{SaltProc v2.0+ architecture}
The SaltProc v2.0+ Python toolkit coupled directly with the Serpent 2 input 
and output files, to allow the reprocessing system couples to depletion 
calculation. Existing PyNE interfaces are employed to parsing Serpent output  
while newly developed interfaces handle input. The standard \gls{OOP} features 
of Python 3 are used to create a flexible, user-friendly tool with increased  
potential for further improvement and collaboration.  
Figure~\ref{fig:saltproc_class} illustrates the SaltProc v2.0+ class 
structure, consisting of 4 main classes:
\begin{figure}[ht!] % replace 't' with 'b' to \centering
  \includegraphics[width=1.07\textwidth]{saltproc_class_diagram.png}
  	  	\vspace{-0.35in}
  \caption{SaltProc v2.0+ python package class diagram in UML notation with  
  examples of object instances.}
  \label{fig:saltproc_class}
\end{figure}
	\paragraph{Depcode.}Contains attributes and methods for 
	reading the user's input file for the depletion software, initial material 
	(e.g., fuel and/or fertile salt) composition, principal parameters for  
	burnup simulation (e.g., neutron population and number of cycles for Monte 
	Carlo neutron transport), and running the depletion code.
	\paragraph{Simulation.} Runs Serpent depletion step, creates and writes 
	HDF5 database, tracks time and converts isotopic composition vector 
	nuclide names from Serpent to human-readable format.
	\paragraph{MaterialFlow.}Each \textit{MaterialFlow} object represents the 
	material flowing between \textit{Process} objects. All instances of this 
	class contain an isotopic composition vector (PyNE Material object  
	initialized from Serpent output file \textbf{dep.m}), mass flow rate, 
	temperature, density, volume, and void fraction. Existing PyNE Material 
	capabilities allow us to easily convert the units of the isotopic 
	composition vector (e.g., from atomic density provided by Serpent to 
	a mass fraction or absolute mass in desired units) and decay the material 
	(i.e. model the \gls{MSBR} protactinium decay tank), calculate decay heat, 
	activity, and dose. The main purpose of the \textit{MaterialFlow} object 
	is to pass detailed information about the salt - starting at the \gls{MSR} 
	vessel outlet - throughout reprocessing components 
	(\textit{Processes}). These processes modify the \textit{MaterialFlow} 
	object before depleting the material in the next Serpent burnup step.
	\paragraph{Process.}Each \textit{Process} object represents a 
	realistic fuel processing step characterized by its throughput rate, 
	volumetric capacity, extraction efficiency for each target element (can be 
	a function of many parameters), waste streams, and other parameters 
	specific to the particular process. Feed \textit{Process} injects fresh 
	fuel salt \textit{MaterialFlow} directly into the reactor core (e.g., 
	adding fissile material with a specific mass flow rate to  
	\textit{MaterialFlow} after performing all removals).

The proposed class structure provides outstanding flexibility in simulating 
various \gls{MSR} fuel processing system designs. A library of various 
\textit{MaterialFlow} (e.g., fuel salt flow, fertile salt flow, refueling salt 
flow) and \textit{Process} (e.g., helium sparging facility, gas separator, 
lanthanide removal component) objects will be created to allow a user to 
quickly create a model of a desired reprocessing scheme. At runtime, the user 
will connect \textit{Process} objects in series or parallel with 
\textit{MaterialFlow} objects to form a comprehensive reprocessing system. The 
user will also be able to create custom objects with desired attributes and 
methods as well as contribute them back to the code package using GitHub 
(https://github.com/ arfc/saltproc).	

\subsection{SaltProc v2.0+ flowchart}
Figure~\ref{fig:saltproc_flow} illustrates the online reprocessing simulation 
algorithm, coupling Serpent with SaltProc v2.0+. To perform a depletion step, 
SaltProc v2.0+ reads a user-defined Serpent template file. The template  
contains input parameters such as geometry, material, isotopic composition, 
neutron population, criticality cycles, total heating power, and boundary 
conditions. SaltProc v2.0+ fills in the template file and runs the Serpent 
single-step depletion. After the depletion calculation, SaltProc v2.0+ reads 
the depleted fuel composition file into the \textit{MaterialFlow} object 
(\textit{core\textunderscore outlet} in Figure~\ref{fig:saltproc_flow}). This 
\textit{MaterialFlow} object contains an isotopic composition vector, total 
volume of material, total mass, mass flow rate, density, temperature, void 
fraction, etc. For the simplest reprocessing case, when all fuel processing 
components are located in-line (100\% of total material flow goes through 
a chain of separation components), the \textit{core\textunderscore outlet} 
object is flowing sequentially between \textit{Processes}, and each 
\textit{Process} is removing a mass fraction of the target elements with 
specified extraction efficiency. Afterward, the removed material mass is 
replenished by fresh fuel salt to maintain the salt inventory in a primary 
loop. Finally, the resulting isotopic composition after reprocessing is stored 
in the HDF5 database and dumped in a new composition file for the next 
Serpent depletion run. SaltProc v2.0+ also stores isotopic composition before 
reprocessing and waste stream from each fuel processing component in a 
database. 
\begin{figure}[ht!] % replace 't' with 'b' to \centering
	\centering
  \includegraphics[width=1.03\textwidth]{saltproc_flowchart.pdf}
  	  	\vspace{-0.35in}
  \caption{SaltProc v2.0+ python package flow chart.}
  \label{fig:saltproc_flow}
\end{figure}

For a more general case with multiple concurrent extraction processes, a 
separate \textit{MaterialFlow} object is created for each branch with a 
user-defined mass flow rate (e.g. 90\% of total mass flow rate flows through 
the left branch and 10\% through the right branch). The total mass and  
isotopic composition vector for each \textit{MaterialFlow} object is 
calculated as a fraction of incoming \textit{core\textunderscore outlet} flow. 
Then, each \textit{MaterialFlow} object is passed through a cascade of 
\textit{Processes} to separate selected chemical elements with a specific 
efficiency. Finally, the \textit{MaterialFlow} object from the left branch is  
merged with the right-hand-side and just like the previous case, a fresh 
fuel salt feed compensates the loss of mass in the separation facilities and 
keeps the fuel salt mass in the primary loop constant.

The class diagram (Figure~\ref{fig:saltproc_class}) allows user to model the 
operation of a complex, multi-zone, multi-fluid \gls{MSR} and is sufficiently 
generalized to represent numerous reactor systems. The refactored version of 
SaltProc stores and edits the isotopic composition of the fuel stream, making  
it a flexible tool to model any geometry: an infinite medium, a unit cell, a 
multi-zone simplified assembly, or a full core. This flexibility allows the 
user to perform simulations of varying fidelity and computational 
intensity. SaltProc v2.0+ is an open-source tool (though having Serpent 
installed is required to use SaltProc v2.0+) available on GitHub. It leverages 
unit and continuous tests crucial for sustainable development 
\cite{krekel_pytest_2004}. It will also have documentation generated through 
Sphinx, a documentation generator, for ease of use \cite{brandl_sphinx_2009}. 
In summary, the development of SaltProc v2.0+ is focused on producing a 
generic, flexible and expandable tool to give the Serpent 2 Monte Carlo code 
the ability to conduct advanced in-reactor fuel cycle analysis as well as 
simulate many online refueling and fuel reprocessing systems.

\section{SaltProc demonstration case}
The SaltProc v2.0+ modeling and simulation tool is demonstrated for the 
\gls{TAP} \gls{MSR} with static core geometry, \gls{LEU} 5\% startup 
composition  \cite{transatomic_power_corporation_neutronics_2016} and the 
three following fueling scenarios: (1) no \gls{FP} removal or feed (Serpent 
only); (2) a 5\% \gls{LEU} online feed; and (3) a 19.79\% \gls{LEU} online 
feed. The primary focus and the bulk of the analysis herein has been on the 
last fueling scenario using 19.79\% \gls{LEU}. All calculations are run with 
Serpent version 2.1.31 and the JEFF-3.1.2 nuclear data library 
\cite{leppanen_serpent_2013, oecd/nea_data_bank_jeff-3.1.2_2014}.

\subsection{Serpent 2 full-core model} \label{sec:tap_model}
The advanced geometric surfaces and transformation capabilities of Serpent 
\cite{leppanen_serpent_2013} are employed to represent \gls{TAP} core. 
Figure~\ref{fig:tap-serpent-plan} shows the $XY$ section of the whole-core
configuration at the expected operational level of the reactor with all
control rods fully withdrawn. Figures~\ref{fig:tap-serpent-elev} and 
~\ref{fig:tap-serpent-elev-zoom} depict a longitudinal section of the reactor. 
This model contains the moderator rods with silicon carbide cladding, 
inlet/outlet plena, and the pressure vessel (Table~\ref{tab:tap_model_param}). 
The fuel salt flows around rectangular moderator assemblies consisting of 
lattices of small-diameter zirconium hydride rods in a corrosion-resistant 
material. The \gls{SVF} in the core is a parameter similar to the widely-used 
moderator-to-fuel ratio and is defined as:
\begin{align}
SVF &= \frac{V_F}{V_F+V_M} = \frac{1}{1+V_M/V_F}
	\intertext{where}
 	V_F &= \mbox{the fuel volume} \nonumber \\
 	V_M &= \mbox{the moderator volume} \nonumber \\
 	V_M/V_F &= \mbox{the moderator-to-fuel salt ratio} \nonumber
\end{align}
The \gls{SVF} for model herein is 0.907268 which means the modeled core is 
under-moderated and has an intermediate spectrum.
\begin{figure}[htp!] % replace 't' with 'b' to 
  \centering
		  \includegraphics[width=\textwidth]{tap_plan_view.png}
  \caption{An $XY$ section of the \gls{TAP} model at the horizontal midplane 
  with fully withdrawn control rods at \gls{BOL} (\gls{SVF}$=0.907268$). 
  The violet color represents zirconium hydride, and the yellow represents 
  fuel salt. The blue color shows Hastelloy-N, the alloy used for the vessel 
  wall, and the white color is the air.}
  \label{fig:tap-serpent-plan}
\end{figure}
\begin{figure}[htp!] % replace 't' with 'b' to 
  \centering
		  \includegraphics[width=\textwidth]{tap_elev_view.png}
		 \vspace{-0.35in}
  \caption{An $XZ$ section of the \gls{TAP} model.}
  \label{fig:tap-serpent-elev}
\end{figure}
\begin{figure}[htp!] % replace 't' with 'b' to 
  \centering
		  \includegraphics[width=0.45\textwidth]{tap_elev_view_zoomed.png}
		 \vspace{-0.2in}
  \caption{Zoomed $XZ$ section of the top of the moderator rods and guide 
  tubes for the \gls{TAP} model. The orange color shows 70–30\% 
  Gd$_2$O$_3$–Al$_2$O$_3$ ceramic absorbers used for the control rods.}
  \label{fig:tap-serpent-elev-zoom}
\end{figure}

To represent the reactivity control system the model has: (1) control rod 
guide tubes made of nickel-based alloy; (2) control rods represented as hollow 
70-30\% Gd$_2$O$_3$-Al$_2$O$_3$ cylinders with a thin Hastelloy-N coating 
\cite{betzler_assessment_2017}; (3) the air inside the guide tubes and 
control rods. The control rod assembly design has yielded a cluster of 25 rods 
that provide a total reactivity worth of 1121pcm\footnote{ 1 pcm = 
10$^{-5}\Delta k_{eff}/k_{eff}$.}.

The control rod cluster is modeled using the \textbf{TRANS} Serpent 2 feature, 
which allows easy change of the control rods position during simulation. All 
figures of the core in this report were generated using the built-in Serpent 
plotter.
%%%%%%%%%%%%%%%%%%%%%%%%%%%%%%%%%%%%%%%%%%%%%%%%%%
\begin{table}[h!]
        \caption{Geometric parameters for the full-core 3D model of 
        \gls{TAP} (reproduced from Betzler \emph{et al.} \cite{betzler_assessment_2017}). }
          \centering
        \begin{tabularx}{0.9\textwidth}{s s x p{0.15\textwidth}}
        \hline
\textbf{Component} & \textbf{Parameter} & Value      		& Unit		             \\ \hline
\multirow{4}{*}{\begin{tabular}[c]{@{}l@{}}Moderator\\ rod\end{tabular}} 
		 & Cladding thickness      	  			    & 0.10 & cm				 \\  
         & Radius 				      	  			& 1.15 & cm				 \\  
         & Length				      	  			& 3.0  & m				 \\  
         & Pitch				      	  			& 3.0  & cm  			 \\ \hline 

\multirow{2}{*}{\begin{tabular}[c]{@{}l@{}}Moderator\\ assembly\end{tabular}} 
         & Array				      	  			& 5 $\times$ 5 & rods$\times$rods \\  
         & Pitch				      	  			& 15.0 & cm    				 \\  \hline

\multirow{4}{*}{\begin{tabular}[c]{@{}l@{}}Core\end{tabular}}          
         & Assemblies  				   	  			& 268  & assemblies/core \\  
         & Inner radius			      	  			& 1.5  & m    				 \\  
         & Plenum height			   	  			& 25.0 & cm    				 \\  
         & Vessel wall thickness     	  			& 5.0 & cm    				 \\ \hline            
        \end{tabularx}
        \label{tab:tap_model_param}
\end{table}
%%%%%%%%%%%%%%%%%%%%%%%%%%%%%%%%%%%%%%%%%%%%%%%%
\subsection{Simulated fuel reprocessing system}
We thoroughly analyzed the original \gls{TAP} reprocessing system design 
(figure~\ref{fig:tap-reproc}) and neutron poisons removal rates 
(table~\ref{tab:reprocessing_list}) to determine a suitable reprocessing 
scheme for the SaltProc v2.0+ demonstration 
(figure~\ref{fig:demo-repro-scheme}).
\begin{figure}[htp!] % replace 't' with 'b' to 
  \centering
		  \includegraphics[width=0.93\textwidth]{demo_reprocessing_scheme.png}
  \caption{\gls{TAP} reprocessing scheme flowchart used for SaltProc v2.0+ 
  demonstration. Arrows represent material flows; percents - fraction of total 
  mass flow rate; ellipses - fuel reprocessing system 
  components; diamonds - waste streams; the box shows refuel material flow.}
  \label{fig:demo-repro-scheme}
\end{figure}

The gas removal components (the sparger and entrainment separator) are located 
in-line because the estimated full loop time for the fuel salt is about 
18 sec and has an approximately equal cycle time 
(table~\ref{tab:reprocessing_list}). 
To remove all volatile gases every 20 sec, the fuel reprocessing system must 
operate with 100\% of the core throughout flow rate and an exceptional 
efficiency. To achieve required cycle time for the demonstration case herein 
we assumed xenon, krypton, and hydrogen extraction efficiencies for the 
sparger and entrainment separator are equal 60\% and 97\%, respectively.

The nickel filter in the \gls{TAP} concept is designed to extract noble metals 
and volatile fluorides. Similarly to volatile gases, noble metals must be 
removed every 20 sec and, hence, the filter should also be able to operate 
in-line. The nickel filter removes a wide range of elements with various 
efficiencies. We calculated these efficiencies for SalProc v2.0+ input 
from removal rates reported in table~\ref{tab:reprocessing_list}.

Lanthanides and other non-noble metals generally have a lower capture 
cross-section and absorb fewer neutrons than gases and noble metals. These 
elements can be removed via a liquid-metal/molten salt extraction process with 
relatively low removal rates (cycle time > 50 days). This is accomplished 
using small fuel salt flow rate (10\% of the core throughout flow rate) via 
liquid-metal/molten salt component, where lanthanides are removed with 
specific extraction efficiency to match required cycle time  
(table~\ref{tab:reprocessing_list}). The remaining 90\% of the flow is 
directed  from the nickel filter to heat exchanger without performing any fuel 
salt treatment.

The removal rates vary among the nuclides in this reactor concept, which 
dictate the necessary resolution of depletion calculations. If the depletion 
time intervals are very short, an enormous number of depletion steps are 
required in order to obtain the equilibrium composition. On the other hand, if 
the depletion calculation time interval is too long, the impact of short-lived 
fission products is not captured. To compromise, a 3-day interval was selected 
based on Betzler \emph{et al.} timestep refinement study  
\cite{betzler_assessment_2017}. For longer, lifetime-long depletion  
simulations, 30-day timestep size will be applied.

\section{Results}
The SaltProc v2.0+ online reprocessing simulation package is demonstrated for 
analyzing \gls{TAP} \gls{MSR} neutronics and fuel cycle to find the equilibrium core composition and core depletion. The neutron population per cycle and the number 
of active/inactive cycles were chosen to obtain a balance between reasonable 
uncertainty for a transport problem (25 pcm for effective multiplication factor) 
and computational time. We accomplished it by setup neutron population 15'000, 
the number of active cycle 400, and the number of inactive cycle 200. 
The \gls{TAP} depletion was performed on 64 Blue Waters 
XE6 nodes (two AMD 6276 Interlagos CPU per node, 16 floating-point Bulldozer 
core units per node or 32 ``integer'' cores per node, nominal clock speed is 
2.45 GHz). The total computational time for calculating the equilibrium 
composition was approximately 9000 node-hours ($\approx$16 core-years).

\subsection{Effective multiplication factor}
Figures~\ref{fig:keff}, \ref{fig:keff-zoomed}, \ref{fig:keff-zoomed-2} demonstrate 
the effective 
multiplication factors  obtained using SaltProc v2.0+ and Serpent. We obtained 
the effective multiplication factors after removing fission products 
and adding feed material at the end of each depletion step (3 days for this 
work). The $k_{eff}$ fluctuates significantly as a result of the batch-wise 
nature of the online reprocessing strategy used.
\begin{figure}[htp!] % replace 't' with 'b' to 
		  \includegraphics[width=1.05\textwidth]{keff_3.png}
	\vspace{-0.2in}
  \caption{Effective multiplication factor dynamics for full-core
   \gls{TAP} model for different fueling scenarios over a 13-year reactor operation. 
   Confidence interval $\pm\sigma=28pcm$ is shaded. Clearly, the reactor went 
   subcritical too fast and further investigation needed to overcome this issue. 
   Possible solutions are: (1) reduce neutron leakage from the core by 
   introducing thick graphite reflector and thermal insulation around 
   vessel to increase effective multiplication factor at the 
   \gls{BOL} to 1.035; (2) extract poisons with faster removal rate;
   (3) use another fissile material for the feed (i.e., \gls{TRU} elements from spent 
   \gls{LWR} fuel); (4) adjust \gls{SVF} on-the-fly by 
   moving moderator assemblies during operation 
   \cite{transatomic_power_corporation_technical_2016} or adding moderator rods 
   only at regular intervals during shutdown for reactor maintenance  \cite{betzler_fuel_2018}.}
  \label{fig:keff}
\end{figure}
\begin{figure}[htp!] % replace 't' with 'b' to 
  \centering
		  \includegraphics[width=0.85\textwidth]{keff_zoomed_1.png}
	  \vspace{-0.25in}
  \caption{Zoomed effective multiplication factor for the first 104 EFPD 
  after startup.}
  \label{fig:keff-zoomed}
\end{figure}
\begin{figure}[htp!] % replace 't' with 'b' to 
  \centering
		  \includegraphics[width=0.85\textwidth]{keff_zoomed_2.png}
	 \vspace{-0.25in}
  \caption{Zoomed effective multiplication factor for the time interval 
  from 367 to 471 EFPD after startup.}
  \label{fig:keff-zoomed-2}
\end{figure}

Loading the initial fuel salt composition with 5\% \gls{LEU} into the 
\gls{TAP} core leads to a supercritical configuration with an excess 
reactivity of about 1900pcm (Figure~\ref{fig:keff}). Without performing any 
fuel salt reprocessing, the core became subcritical after 30 days of operation 
(Figure~\ref{fig:keff-zoomed}). 
We obtained this result using Serpent ONLY without introducing any \gls{FP} 
extraction and refueling. For the beginning of the \gls{TAP} lifetime, uranium 
enrichment in the feed has a minor effect because the tiny amount of poisons 
was produced (<1kg/day) and, hence, a small mass of fresh salt was injected. 
Notably, the core went subcritical after 42 days of operation with either  
\gls{LEU} 5\% or \gls{LEU} 19.79\% feed.

The \gls{TAP} core never reached equilibrium fuel salt composition without 
performing fuel salt reprocessing and refueling. For the fueling scenarios 
with 5\% and 19.79\% \gls{LEU} feed, the reactor achieved the equilibrium 
state after 10 years of operation. Overall, the effective multiplication 
factor gradually decreases from the initial 1.018 to 0.88 for the 19.79\% 
\gls{LEU} feed and 0.86 for the 5\% \gls{LEU} feed, which indicates 
problems with operating this nuclear reactor design. We will try to overcome 
this issue by re-optimizing the \gls{TAP} core and design parameters as well 
as adding new functionality to SaltProc v2.0+.

Acting as a complement to Figure ~\ref{fig:keff}, the Figure~\ref{fig:shannon} 
shows the Shannon entropy of a fission source as a function of the number of 
inactive cycles and clearly indicates that the Monte Carlo simulation 
converge with a number of inactive cycles >200 
\cite{brown_k-effective_2011-1}. 
\begin{figure}[htp!] % replace 't' with 'b' to 
  \centering
		  \includegraphics[width=\textwidth]{h_src.png}
	 \vspace{-0.35in}
  \caption{Shannon entropy of a fission source for initial and equilibrium 
  fuel salt composition (19.87\% \gls{LEU} feed) as a function of inactive 
  cycles number for the full core calculations with neutron population $M=15'000$.}
  \label{fig:shannon}
\end{figure}

\subsection{Neutron spectrum}
Figure~\ref{fig:spectrum} shows the normalized neutron flux spectrum for the 
full-core \gls{TAP} core model in the energy range from 10$^{-8}$ to 15 MeV. 
The neutron energy spectrum at equilibrium is a little bit harder than at 
startup due to the accumulation of plutonium and other strong absorbers in the 
core during reactor operation. The \gls{TAP} spectrum is significantly 
harder than in a typical \gls{LWR} and is in a good agreement with 
\gls{ORNL} report \cite{betzler_assessment_2017}.
\begin{figure}[htp!] % replace 't' with 'b' to 
  \centering
		  \includegraphics[width=\textwidth]{spectrum.png}
	 \vspace{-0.35in}
  \caption{The neutron flux energy spectrum normalized by unit lethargy 
  for initial and equilibrium fuel salt composition.}
  \label{fig:spectrum}
\end{figure}
\subsection{Fuel salt composition}
Figure~\ref{fig:u-pu} shows the absolute mass of major heavy isotopes 
which have a strong influence on the reactor core physics. The mass of 
$^{236}$U, $^{238}$U, $^{239}$Pu, $^{240}$Pu, and $^{241}$Pu in the 
fuel salt changes insignificantly after approximately 10 years of operation,
which matches stabilization time for the effective multiplication factor. 
Hence, the quasi-equilibrium state was reached after 10 years of reactor 
operation. Moreover, the \gls{TAP} core bred approximately the same amount 
of fissile $^{239}$Pu ($\approx2$t) as was initial fissile material 
($^{235}$U) load. A significant amount of non-fissile plutonium builds 
up during operation and accounts for 50\% of the plutonium after 13 years 
of operation. Overall, the rate of breeding fissile $^{239}$Pu from $^{238}$U 
even in a relatively hard neutron spectrum is not sufficient to compensate for 
the negative effects of strong absorber accumulation to maintain the reactor 
critical.
\begin{figure}[htp!] % replace 't' with 'b' to 
  \centering
		  \includegraphics[width=1.03\textwidth]{u_pu_mass.png}
	 \vspace{-0.4in}
  \caption{Mass of major nuclides during 13 years of reactor operation 
  with 19.79\% \gls{LEU} feed.}
  \label{fig:u-pu}
\end{figure}

We checked the correctness of SaltProc v2.0+ by comparing the mass of the 
important isotopes ($^{135}$Xe, $^{135}$I) for load-following operation to an 
expected mass after each depletion step (Figure~\ref{fig:xe-i}). The expected 
mass of a $^{135}$Xe was calculated as follows:
\begin{align}
& m_{after\;reprocessing} = m_{before\;reprocessing} \times  \epsilon_{sparger} \times \epsilon_{separator}
	\intertext{where}
 	m_{after} &= \mbox{the mass of the isotope after applying removals and feeds} \nonumber \\
 	m_{before} &= \mbox{the mass of the isotope right before  reprocessing} \nonumber \\
 	\epsilon_{sparger} &= \mbox{the sparger extraction efficiency} \nonumber \\
 	\epsilon_{separator} &= \mbox{the entrainment separator extraction efficiency} \nonumber
\end{align}
\begin{figure}[htp!] % replace 't' with 'b' to 
  \centering
		  \includegraphics[width=\textwidth]{xe_i_mass.png}
	 \vspace{-0.35in}
  \caption{Mass of major neutron poison, $^{135}$Xe, and its main precursor, 
   $^{135}$I, during 13 years of reactor operation before and after reprocessing.}
  \label{fig:xe-i}
\end{figure}

The $^{135}$I approach is similar, but the extraction efficiency of iodine in 
the nickel filter is only 5\%. Figure~\ref{fig:xe-i} shows that SaltProc v2.0+ 
extraction module correctly removes target isotopes with the specified 
extraction efficiency: SaltProc and expected mass match. Overall, the 
\gls{TAP} fuel reprocessing system simulated with SaltProc v2.0+ allows 
maintaining $^{135}$Xe inventory in the core as low as 1g during operation on 
100\% power.

\section{Future work}
The \gls{TAP} core should be able to maintain a critical state ($k_{eff}\geq 
1.0$) for at least 30 years of operation lifetime. We will re-optimize and 
improve the \gls{TAP} reactor model by performing the next steps:
\paragraph{$k$ eigenvalue at \gls{BOL}:} The effective multiplication factor 
is too small at the \gls{BOL}. The most recent \gls{ORNL} paper  
\cite{betzler_fuel_2018} reported the initial $k$ eigenvalue calculated 
for \gls{BOL} to be about 1.035, much greater than our result 
($1.01909\pm23pcm$). We will reduce fast neutron leakage by adding an 
appropriate reflector and thermal insulation around the vessel to reach a 
larger excess of reactivity at the \gls{BOL}.
\paragraph{Dynamic moderator-to-fuel ratio:} The notable feature of the 
\gls{TAP} is the ability to adjust moderator-to-volume, or \gls{SVF}, ratio 
during lifetime by changing the moderator rods configuration. Adding more 
moderator to the core thermalizes the neutron spectrum and significantly 
extends the core lifetime. Unfortunately, the \gls{TAP} White 
papers and \gls{ORNL} technical reports lack details about how those 
configurations are formed. We will create various geometries with various 
\gls{SVF} based on the assumption, that the plant personnel is reconfiguring 
the moderator rods only at regular intervals (i.e., 18 months) during the 
shutdown for reactor maintenance. That is, we assume that the reactor 
maintaining the long-term reactivity by periodically replacing stationary
zirconium hydride rod assemblies with those containing more rods (e.g., 
replacement of a four-rod assembly with a nine-rod assembly)  
\cite{betzler_fuel_2018}. Additionally, we will add in a SaltProc v2.0+ 
capability to switch from one geometry file to another with a user-defined 
time interval. 
\paragraph{Reprocessing scheme:} Extraction efficiencies and refueling 
strategies of the \gls{TAP} fuel reprocessing and refueling plant will be 
revised to ensure that all possible strong poisons are removed at an 
appropriate rate.

\bibliographystyle{ieeetr}
\bibliography{q4-report}

%\bibliographystyle{plain}

%\printbibliography

\end{document}
