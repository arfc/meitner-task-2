\section{Introduction}

    The University of Illinois, Urbana-Champaign (UIUC) undertook a series of studies to demonstrate a fuel processing system that enables load-following in Molten Salt Reactors (MSRs). Thus far, we demonstrated and released the online fuel salt processing tool (Saltproc V0.2 \cite{rykhlevskii_saltproc_2018}) for Molten Salt Reactors per Milestones 2.1 and 2.2. This report presents the progress we have made towards Milestone 2.3.

    This study performs sensitivity analysis with the SaltProc during different load-follow transients to decide the fuel reprocessing system design and assess the load-following capabilities of the MSRs. The details of the used method, simulation codes, sparging system design considerations, results of load-follow simulation with SaltProc, and results of sensitivity analysis for the sparging system were provided.

    This study considers Molten Salt Breeder Reactor (MSBR) \cite{robertson_conceptual_1971} design as the results of the previous reports \cite{rykhlevskii_milestone_2019} show that the TAP MSR is able to operate at load-follow transients without a gas removal system. Those reports also clearly say that "The gas removal system is not necessary to ensure safe TAP system operation during a short-term power drop and restart transient".

    To devise a gas removal system (sparging) design during the load-following operations, we followed three steps:
    \begin{itemize}
        \item Sensitivity analysis for reactor core at different load-follow scenarios when the sparging system is activated.
        \item Sensitivity analysis for the sparging system to determine the boundaries of its design parameters.
        \item Assessment and determination of design parameters from the results of the first two steps.
    \end{itemize}

    What we performed in this milestone are: (i) to evaluate realistic load profile with $\geq$ $\pm$ 10\% capacity/min reactor power ramping, (ii) to make sensitivity analysis by varying removal efficiency in a wide range (w/ Task 1), (iii) to find out smart gas removal regulation bounding.

\section{Milestone objectives}

    The finalized work plan for this project (DOE ARPA-E MEITNER award DE-AR0000983) formulated the goal of Milestone 2.3 as follows:

    \begin{quotation}
        "Recommend fuel processing system design (and feasible design
        space) that can achieve Xe removal for $\ge \pm 10\%$ capacity/min reactor power ramping; approved by ARPA-E."
    \end{quotation}

    This milestone has been completed through the addition of a sparging system package to the existing Saltproc version. In this document, we will discuss the results of the sensitivity analysis and demonstrate a prototype design for the Xe removal system, concerning design and safety margins, as well as key parameters for improved performance.

\section{Methods}

\subsection{MSBR Reactor}

    We considered the thorium-fueled MSBR design \cite{robertson_conceptual_1971} developed by ORNL due to the reason described in the "Introduction" section. Reactor model was elaborated in Rykhlevskii \emph{et al.} \cite{rykhlevskii_modeling_2019}.

\subsection{System Codes}

\subsubsection{Saltproc}

    We used the Saltproc \cite{rykhlevskii_saltproc_2018} for online reprocessing system modeling. The tool, as shown in Figure \ref{fig:scheme}, was developed for online reprocessing system modeling and demonstrated/validated for TAP and MSBR designs. It has a full capability of simulating fueling strategy and core geometry changes during operation. The SaltProc provides fuel composition to reactor multiphysics task. Details of the code were given in Milestone 2.1 Report \cite{rykhlevskii_milestone_2019}.

    \begin{figure}[h]
        \begin{center}
            \includegraphics[width=0.6\textwidth]{principal_scheme.png}
        \end{center}
        \caption{Principal scheme of xenon removal from the salt.}
        \label{fig:scheme}
    \end{figure}

\subsubsection{Dakota}

    We used the Dakota code \cite{adams_dakota_2019} for sensitivity analysis to assess the reactor performance at different load-follow scenarios when the sparging system is enabled. The code provides optimization tools including sensitivity analysis packages for miscellaneous design problems. With Dakota group's words:

    \begin{quote}
        "In addition to its state-of-the-art optimization methods, Dakota
        includes methods for global sensitivity and variance analysis, parameter
        estimation, uncertainty quantification, and verification, as well as
        meta-level strategies for surrogate-based optimization, hybrid
        optimization, and optimization under uncertainty."
    \end{quote}

    We performed a multidimensional parameter study by computing response data sets for an n-dimensional hypergrid formed by sensitivity parameters. Each sensitivity parameter is partitioned into equally-spaced intervals between its upper and lower bounds. Each combination was then simulated by Saltproc software coupled to Serpent.

\subsubsection{Serpent}

    Serpent software \cite{Lep2014} was used for Monte-Carlo based neutron transport calculations. Serpent model with corresponding Serpent's parameters stands on the MSBR design described in various reports prepared for ARPA-E MEITNER Program \cite{rykhlevskii_fuel_2019, rykhlevskii_modeling_2019, rykhlevskii_fuel_2020}.

\newpage
\FloatBarrier

\section{Sparging Design}

    A sparging system is composed of two separate components: Sparger (bubble generator) and Separator (bubble separator). The role of the sparger is to generate He bubbles in which noble gases diffuse while the role of the separator is to remove bubbles from fuel salt which carry volatile noble gases like Xe and Kr. The efficiency of the sparger component is expressed in terms of gas removal efficiency ($\epsilon_{X}$) while that of the separator is in terms of the bubble separation efficiency ($\epsilon_{sep}$). Accordingly, total gas removal efficiency becomes ${\epsilon^{X}}_{total} = \epsilon_{X} \times \epsilon_{sep}$ for any target isotope $X$. A simple illustration of the system is given in Figure \ref{fig:sparging}.

    \begin{figure}[htbp!]
        \begin{center}
            \includegraphics[width=0.8\textwidth]{bubble_separator_main.png}
        \end{center}
        \caption{Fission product removal system: Sparging}
        \label{fig:sparging}
    \end{figure}

\subsection{Sparger Design}

 As illustrated in Figure \ref{fig:sparging}, sparger includes a contactor section and a long pipe with a contactor cross-section of $A_c = \pi\times d^2/4$ and length of $L$. Gas removal efficiency in bubble generator is defined by equation in Figure \ref{fig:eq2} based on \cite{peebles_removal_1968}.

    \begin{figure}[h!]
        \begin{center}
            \includegraphics[trim={0 0 30 30}, clip, width=1.0\textwidth]{eq2.1-part1.png}
            \includegraphics[width=0.7\textwidth]{eq2.1-part2.png}
        \end{center}
        \caption{Removal efficiency correlation from \cite{peebles_removal_1968}}
        \label{fig:eq2}
    \end{figure}

    In the equation, gas-liquid interfacial area per unit volume ($a$) is:

    \begin{equation}\label{interfacial}
        a = \frac{6}{d_b} \frac{Q_{He}}{Q_{salt}+Q_{He}}
    \end{equation}

    Also, the universal gas constant is 8.314 L.Pa/mol-K, and Henry's law constant is at the operating salt temperature. Gas constants for solute gases of Xe, Kr, and H are then calculated at the operating salt temperature based on a reference temperature of 298.15 K with the following expression \cite{acp-15-4399-2015}:

    \begin{equation}\label{henry}
        H(T) = H(T_{ref})\times\exp(C(\frac{1}{T}-\frac{1}{T_{ref}}))
    \end{equation}
    where C denotes the exponential constant and the constants for Xe, Kr, and H elements are 2300, 1900, and 0 K, respectively. Henry's law constants for Xe, Kr, and H elements at the reference temperature are  4.3e-5, 2.5e-5, and 2.6e-6 Pa/mol-L, respectively \cite{acp-15-4399-2015}.

    Liquid phase mass transfer coefficient ($K_L$) determined by the flow dynamics is calculated by the following formula:

    \begin{equation}\label{kl}
        K_L = Sh \times D / d_p
    \end{equation}
    where $D$ is the liquid phase diffusivity of 2.5e-9 (cm$^2$/s) (from CFD group), $Sh$ is the Sherwood number, and $d_p$ is the pipe diameter. The dimensionless $Sh$ number developed in Milestone 1.2 is defined as follows:

    \begin{equation}\label{sh}
        Sh = 2.06972 * Re_D^{0.555} * Sc^{0.5}
    \end{equation}
    where $Re_D$ is the pipe Reynolds number and $Sc$ is the dimensionless Schmidt number defined in \ref{sc}:

    \begin{equation}\label{sc}
        Sc = \nu/D
    \end{equation}
    where $\nu$ is the kinematic viscosity ($\mu/\rho$) in m$^2$/s. Here, for the calculation of the pipe Reynolds number, we used

    \begin{equation}\label{reynold}
        Re_D = \frac{Dv}{\nu}
    \end{equation}
    where $v$ denotes the fluid velocity in m/s. Temperature-dependent density and dynamic viscosity of the fluid were supplied by the CFD group and are defined in Eq. \ref{density} and \ref{viscosity}:

    \begin{equation}\label{density}
        \rho = 6.105 - 0.001272 * T [kg/m^3]
    \end{equation}
    and
    \begin{equation}\label{viscosity}
        \mu = 1.076111581E-2 * (T / 1000)**(-4.833549134) [N.s/m^2]
    \end{equation}
    where T is the salt temperature in Kelvin.

    In a sparger model, the parameters need to be designed are $Q_{salt}$, $Q_{He}$, $L$, $d_p$, $d_b$, and $T_{salt}$.

\newpage
\FloatBarrier

\subsection{Separator Design}

    A detailed design of the separator is given in Figure \ref{fig:bubble_sprt}. We employed the regression model for the estimation of the bubble separation efficiency developed within the scope of Milestone 1.2.

    The model given in Figure \ref{fig:reg_model} is expressed in terms of gas outlet diameter ($D_o$), sparger (pipe) diameter ($D$), bubble diameter ($d_b$), pressure difference ($\Delta p$) between the inlet and the gas outlet, liquid superficial velocity ($j_l$), salt density ($\rho$), nominal void fraction ($\alpha$ = $j_g/(j_{l}+j_{g})$ or $Q_{He}/(Q_{salt}+Q_{He})$), slope of the initial swirling ($k$), cone diameter of the recovery vane ($D_c$ = 3.41 $D_o$), and the pipe Reynolds number ($Re_D$) calculated by Eq. \ref{reynold}. Liquid superfical velocity ($j_l$) is calculated from volumetric salt flow rate in the following way: $j_l = Q_{salt}/A$ where $A = \pi\times D^2/4$ is the contactor cross-sectional area.

    In a separator model, the parameters need to be designed are $Q_{salt}$, $Q_{He}$, $D_o$, $d_p$, $d_b$, $\Delta p$, and $T_{salt}$.

    \begin{figure}[htbp!]
        \begin{center}
            \includegraphics[width=\textwidth]{bubble_separator_detailed.png}
        \end{center}
        \caption{Entrainment Separator}
        \label{fig:bubble_sprt}
    \end{figure}

    \begin{figure}[htbp!]
        \begin{center}
            \includegraphics[width=1.1\textwidth]{sep_eff_eq_1.png}
            \includegraphics[width=0.7\textwidth]{sep_eff_eq_2.png}
        \end{center}
        \caption{Regression model from Milestone 1.2 report}
        \label{fig:reg_model}
    \end{figure}

\newpage
\FloatBarrier

\subsection{Integration to Saltproc}

    Sparging system was embedded to Saltproc by separately defining Sparger and Separator clasesses. \textit{read\_processes\_from\_input} function in \textit{app.py} script calls the classes to calculate removal efficiencies of target elements. More information about Saltproc functions and classes can be found in Milestone 2.1 report \cite{rykhlevskii_milestone_2019} and ARFC Github repo (https://github.com/arfc/saltproc). Sparger class uses the equation in Figure \ref{fig:eq2} whereas Separator uses the equation in Figure \ref{fig:reg_model}. In this way, besides the existing flexibility, saltproc now enables Sparging system when the "self" input key is used in the json object input file.

    Later, these equations are to be replaced with improved ones as the experimental and simulation data are supplied by the project's research groups.

\newpage
\FloatBarrier

\section{Sensitivity Analysis}

    To design a feasible sparging system, we performed two separate sensitivity analyses:

    \begin{enumerate}
        \item Reactor core behavior at different load-follow transients and decision on how much removal efficiency is needed to maintain the criticality in the MSBR when the sparging system is enabled.
        \item Specification and optimization of design boundaries for the sparger and separator.
    \end{enumerate}

    Results of these analyses were later combined to understand the ultimate design boundaries of sparging system.

\subsection{Load-Follow Transients}

    Figure \ref{fig:workflow} illustrates a simplified workflow of the sensitivity analysis for the assessment of the reactor core behavior. Results define total $\varepsilon$$_{Xe}$ requirements for prototype Xe sparger and entrainment separator system.

    \begin{figure}[htbp!]
        \begin{center}
            \includegraphics[width=0.35\textwidth]{workflow.png}
        \end{center}
        \caption{Sensitivity analysis workflow.}
        \label{fig:workflow}
    \end{figure}

\subsubsection{Scenarios}

    Two critical load following scenarios were investigated:
    \begin{itemize}
        \item The first worst-case scenario simulates 8 hours at full power after an 8-hour shutdown, providing maximum Xe poisoning effect in the reactor.
        \item The second worst-case scenario considers a short period load-follow for maximum Xe accumulation over time. In this scenario, the reactor runs at full power for one hour after an hour of shutdown, and this repeats several times.
    \end{itemize}

\subsubsection{Sensitivity Parameters}

    We considered the Xe removal efficiency ($\varepsilon$$_{Xe}$) given in Figure \ref{fig:eq2} and the bubble separation efficiency given in Figure \ref{fig:reg_model}, ranging from 0 to 100 \%, as input variables to the Saltproc code. We used k$_{eff}$, $\beta$$_{eff}$, breeding ($\gamma$) and reactivity feedbacks ($\alpha$) as performance metrics. As the development of an experimental correlation to define the bubble separation efficiency of the separator was continued by the CFD group at the moment when we performed this study, we assumed it as 95\%. Note that this value is expected to be between 95 and 100\%.

    Other parameters used in Figure \ref{fig:eq2} to calculate the removal efficiency are as follow: length ($L$) = 11 m, diameter ($d_p$) = 0.4 m, volume ($V$) = 1.4 m$^{3}$, A$_c$ = 0.126 m$^{2}$, He bubble diameter, d$_b$ = 0.508 mm, salt volumetric flow rate ($Q_{salt}$) = 2 m$^{3}$/s, sparging gas (helium) volumetric flow rate ($Q_{He}$) = 0.1 m$^{3}$/s.

\subsection{Sparging System}

    As a final step of the sensitivity analysis for the sparging system, we bounded sparging system design based on the load-following results. For bubble generator (sparger), we explored the effect of sensitivity (design) parameters on gas removal efficiencies of Xe, Kr, and H target elements. In the case of spearator, we considered the bubble separator efficiency.

    To understand the interdependencies of design parameters, we examined individual and binary effects of sparger design parameters on the Xe removal efficiency.

\subsubsection{Sensitivity Parameters}

    Design parameters for sparging system were considered as salt volumetric flow rate ($Q_{salt}$), helium volumetric flow rate ($Q_{He}$), bubble diameter ($d_b$), pipe diameter ($d_p$), pipe length ($L$), pressure difference ($\Delta p$), gas outlet diameter ($D_o$), and salt temperature ($T_{salt}$). The metrics corresponding to the performance of the system are gas removal efficiencies ($\varepsilon$$_{X}$) of the sparger where $_{X}$ denotes Xe, Kr, and H target elements and the bubble separation efficiency of the separator ($\varepsilon$$_{sep}$).

    Base design parameters (supplied by the CFD group) used for comparison in the sensitivity analysis are as follow: pipe length ($L$) = 10 m, pipe diameter ($d_p$) = 0.1 m, bubble diameter ($d_b$) = 1 mm, salt volumetric flow rate ($Q_{salt}$) = 0.1 m$^{3}$/s, salt temperature ($T_{salt}$) = 900 K, pressure difference ($\Delta p$) = 4e5 Pa, gas outlet diameter ($D_o$) = 0.02, and helium volumetric flow rate ($Q_{He}$ = 5\% of $Q_{salt}$) = 0.005 m$^{3}$/s. Accordingly, we changed these parameters between -50\% and +50\% (i.e., $\pm$10\%, $\pm$25\% and $\pm$50\%).

\newpage
\FloatBarrier

\section{Results}

\subsection{Load-Follow}

    We first carried out sensitivity analysis for different load-follow transients. Initial results shown in Figure \ref{fig:loadfollow} and the results from the previous work \cite{rykhlevskii_fuel_2020} point out that MSBR cannot do load-follow without gas removal at BOL (30 days), MOL (15 years), and EOL (30 years) as the effective multiplication factor decreases with the start of the shutdown. Online gas removal from the fuel salt even with moderate efficiency significantly reduces the xenon poisoning effect, yet very high removal efficiency seems unnecessary to negate the negative effect of xenon poisoning. Load-follow at EOL is the worst for k$_{eff}$ and consequently considered for sensitivity analysis.

    \begin{figure}[h]
        \begin{center}
            \includegraphics[width=1.0\textwidth]{msbr_loadfollow.png}
        \end{center}
        \caption{Load follow is attempted at BOL (30 days), MOL (15 years) and EOL (30 years) without gas removal system. Uncertainty in k$_{eff}$ is 25 pcm. 30 mins time resolution.}
        \label{fig:loadfollow}
    \end{figure}

\subsubsection{First Scenario}

    Figure \ref{fig:single_keff} shows the results of the first scenario (single load-follow) for k$_{eff}$. After a lifetime of operation at $\varepsilon$$_{Xe}$ = 0.536, single load-follow was attempted. In this transient, for the base case geometry, the reactor can recover from the Xe poisoning effect after $\varepsilon$$_{Xe}$ = 26.8\%. Generally, increasing gas removal efficiency increases excess reactivity. If higher efficiency is used, then the reactor recovers excess reactivity quicker, within a few hours.

    As to the breeding ratio depicted in Figure \ref{fig:single_breed}, single load-follow transient results in a gradual decrease. Increasing the gas removal efficiency slightly lowers the breeding ratio during the load-follow.

    For the delayed neutron fraction ($\beta$$_{eff}$) given in Figure \ref{fig:single_delayed}, we observed no significant change. Instead, $\beta$$_{eff}$ fluctuates in a narrow range due to the statistical deviation.

    We also explored multiple consecutive load-follow transients causing sharp changes in salt composition. As can be clearly seen in Figure \ref{fig:double_keff}, k$_{eff}$ begins fluctuating with Xe buildup and burndown period. We understand from the result that to keep the reactor stable, gas removal efficiency at least $\varepsilon$$_{Xe}$ = 53.6\% is required (corresponds in base case geometry to K$_{L}$ $>$ 25 ft/hr = 2.117 mm/s).

    With the same load-follow period, we increased the number of transients. We saw from Figure \ref{fig:triple_keff} that a higher gas removal efficiency (at least $\varepsilon$$_{Xe}$ = 76.9\% or K$_{L}$ $>$ 50 ft/hr = 4.233 mm/s) is needed to keep the reactor stable. Therefore, these results indicated that as the number of power ramps increases, a higher gas removal efficiency is required for stable reactor behavior.

    \begin{figure}[htbp!]
        \begin{center}
            \includegraphics[width=0.8\textwidth]{single_ramp_keff.png}
        \end{center}
        \caption{After a lifetime of operation at $\varepsilon$$_{Xe}$= 0.536, load follow is attempted at EOL. Above shows k$_{eff}$ during load follow transient for various total Xe removal efficiencies
        ($\varepsilon$$_{Xe}$) over time after shutdown.}
        \label{fig:single_keff}
    \end{figure}

    \begin{figure}[htbp!]
        \begin{center}
            \includegraphics[width=0.8\textwidth]{single_ramp_breeding.png}
        \end{center}
        \caption{After a lifetime of operation at $\varepsilon$$_{Xe}$= 0.536, load follow is attempted at EOL. Above shows breeding ratio during load
        follow transient for various total Xe removal efficiencies
        ($\varepsilon$$_{Xe}$) over time after shutdown.}
        \label{fig:single_breed}
    \end{figure}

    \begin{figure}[htbp!]
        \begin{center}
            \includegraphics[width=0.8\textwidth]{single_ramp_delayed.png}
        \end{center}
        \caption{After a lifetime of operation at $\varepsilon$$_{Xe}$= 0.536, load follow is attempted at EOL. Above shows $\beta$$_{eff}$ during load
        follow transient for various total Xe removal efficiencies
        ($\varepsilon$$_{Xe}$) over time after shutdown.}
        \label{fig:single_delayed}
    \end{figure}

    \begin{figure}[htbp!]
        \begin{center}
            \includegraphics[width=0.8\textwidth]{double_ramp_keff.png}
        \end{center}
        \caption{After a lifetime of operation at $\varepsilon$$_{Xe}$= 0.536, load follow is attempted at EOL. Above shows k$_{eff}$ during multiple load follow transient for various total Xe removal efficiencies
        ($\varepsilon$$_{Xe}$) over time after shutdown.}
        \label{fig:double_keff}
    \end{figure}

    \begin{figure}[htbp!]
        \begin{center}
            \includegraphics[width=0.8\textwidth]{triple_ramp_keff.png}
        \end{center}
        \caption{After a lifetime of operation at $\varepsilon$$_{Xe}$= 0.536, load follow is attempted at EOL. Above shows  k$_{eff}$ during multiple load follow transient for various total Xe removal efficiencies
        ($\varepsilon$$_{Xe}$) over time after shutdown.}
        \label{fig:triple_keff}
    \end{figure}

\newpage
\FloatBarrier

\subsubsection{Second Scenario}

    Unlike the previous load-follow transients, in this part, we examined short period load-follow transients (second scenario) and implemented four consecutive power ramps. As can be seen in Figure \ref{fig:quadro_keff}, we saw a quick recovery from shutdown even with low gas removal efficiency.

    \begin{figure}[htbp!]
        \begin{center}
            \includegraphics[width=0.9\textwidth]{quadro_ramp_keff.png}
        \end{center}
        \caption{After a lifetime of operation at $\varepsilon$$_{Xe}$= 0.536, load follow is attempted at EOL. Above shows  k$_{eff}$ during multiple load follow transient for various total Xe removal efficiencies
        ($\varepsilon$$_{Xe}$) over time after shutdown.}
        \label{fig:quadro_keff}
    \end{figure}

\newpage
\FloatBarrier

\subsection{Sparging System}

    The results of the previous section directly pointed to a sparging system design adaptable to different load-follow scenarios. This system design should cover a wide range of efficiency as high as 80\% (corresponds to the triple load-follow). We will, therefore, separately handle the sparger and separator designs in this context.

\subsubsection{Sparger Design}

    First of all, we sought potential sparger designs around the base design and changed the considered parameters of the base design up to $\pm$ 50\%. Figure \ref{fig:individual_eff_sparger} and \ref{fig:binary_eff_sparger} show change in Xe removal efficiency. From the binary effects, we can say that each design parameter is acting independently on the removal efficiency. This is because the sub-plots indicate a non-convex structure and any parameter (in x and y axes) on any subplot either always increases or decreases the removal efficiency.

    When we look at the individual effects on sparger design, gas removal efficiencies decrease with increasing bubble diameter and salt flow rate whereas gas removal efficiencies increase with increasing pipe diameter, pipe length, helium flow rate, and salt temperature.

    In addition, due to the different thermo-dynamics and chemical properties of the target elements, the change of tritium removal efficiency somewhat different from that of Xe and Kr removal efficiency, being affected significantly by salt temperature change.

    Furthermore, salt temperature and sparger pipe diameter look like not effective players in the adjustment of the gas removal efficiency as the 10\% increase in temperature only results in about 3\% increase in efficiency. Even with 50\% change, there is only a 10\% gain in efficiency, by increasing from 40\% to about 45\%. In addition, we may need an additional heater before sparger to elevate the salt temperature as the salt temperature is determined by the reactor design itself.

    \begin{figure}[htbp!]
        \begin{center}
            \includegraphics[width=0.45\textwidth]{Sparger_Xe_eff_vs_db.png}
            \includegraphics[width=0.45\textwidth]{Sparger_Xe_eff_vs_dp.png}
            \includegraphics[width=0.45\textwidth]{Sparger_Xe_eff_vs_length.png}
            \includegraphics[width=0.45\textwidth]{Sparger_Xe_eff_vs_Q_salt.png}
            \includegraphics[width=0.45\textwidth]{Sparger_Xe_eff_vs_Q_He.png}
            \includegraphics[width=0.45\textwidth]{Sparger_Xe_eff_vs_temp_salt.png}
        \end{center}
        \caption{Individual effect of design parameters on the Xe removal efficiency ($\varepsilon$$_{Xe}$).}
        \label{fig:individual_eff_sparger}
    \end{figure}

    \begin{figure}[htbp!]
        \begin{center}
            \includegraphics[width=0.8\textwidth]{Sparger_result.png}
        \end{center}
        \caption{Binary effect of design parameters on the Xe removal efficiency ($\varepsilon$$_{Xe}$).}
        \label{fig:binary_eff_sparger}
    \end{figure}

\newpage
\FloatBarrier

\subsubsection{Separator Design}

    Similar to the sparger case, we sought potential separator designs around its base design and changed the considered parameters of the base design up to $\pm$ 50\%. Figure \ref{fig:individual_eff_separator} and \ref{fig:binary_eff_separator} show the effects of these parameters on the bubble separation efficiency. Note that since the presented efficiency is associated with the separation of bubbles from the salt (not with the transfer of target elements), all target isotopes must have the same value.

    From the binary effects, as in the sparger case, all parameters are independent variables. The only convex-like structure is seen in the subplot of average salt temperature versus pipe diameter. But, we think this is because these two parameters are the most critical players on efficiency and their effects are at great amounts when their values are changed slightly.

    According to the results of individual effects, gas removal efficiency increases with increasing salt flow rate, helium flow rate, gas outlet diameter, bubble diameter, and salt temperature whereas it decreases with increasing pressure difference and pipe diameter.

    Among these parameters, the salt temperature cannot be lower than 800 K due to the solidification concerns on the salt at low temperatures.

    Pressure difference and helium flow rate parameters seem insignificant on the separation efficiency. The most important player is the pipe diameter that directly affects the flow dynamics.

    As discussed in Milestone 1.2 for the effect of bubble diameter on the separator, bubble diameter has to be around 1 mm because both sparger and separator are influenced by it. The smaller the bubble diameter the sparger has, the higher gas removal efficiency the sparger yields. Oppositely, the smaller bubble diameter the separator has, the lower bubble separation efficiency the separator yields.

    In any case, we stayed above 95\% bubble separation efficiency while the base design yields about 99\% efficiency.

\begin{figure}[htbp!]
    \begin{center}
        \includegraphics[width=0.4\textwidth]{Separator_sep_eff_vs_db.png}
        \includegraphics[width=0.4\textwidth]{Separator_sep_eff_vs_deltap.png}
        \includegraphics[width=0.4\textwidth]{Separator_sep_eff_vs_do.png}
        \includegraphics[width=0.4\textwidth]{Separator_sep_eff_vs_dp.png}
        \includegraphics[width=0.4\textwidth]{Separator_sep_eff_vs_q_he.png}
        \includegraphics[width=0.4\textwidth]{Separator_sep_eff_vs_q_salt.png}
        \includegraphics[width=0.4\textwidth]{Separator_sep_eff_vs_temp_salt.png}
    \end{center}
    \caption{Individual effect of design parameters on the bubble separation efficiency ($\varepsilon$$_{sep}$).}
    \label{fig:individual_eff_separator}
\end{figure}

\begin{figure}[htbp!]
    \begin{center}
        \includegraphics[width=0.7\textwidth]{Separator_result.png}
    \end{center}
    \caption{Binary effect of design parameters on the bubble separation
    efficiency ($\varepsilon$$_{sep}$).}
    \label{fig:binary_eff_separator}
\end{figure}

\newpage
\FloatBarrier

\section{Conclusion}

    In short, MSBR can, without difficulty, operate under load follow transient with low gas removal efficiency, unless the shutdown period is too long, typically greater than 4 hours. Recovery time depends directly on gas removal efficiency and load-follow period. From the results of the first part, we would need an adjustable sparger/separator design to achieve sufficient gas removal for all kinds of load-follow scenarios.

    From the binary effects of the design parameters of sparger and separator, each design parameter has no relation with the other parameters, acting like a free variable and affecting the removal/separation efficiency independently.

    We can use the separator base design which yields 98.7\% efficiency for the pipe diameter ($d_p$) = 0.1 m, bubble diameter ($d_b$) = 1 mm, salt volumetric flow rate ($Q_{salt}$) = 0.1 m$^{3}$/s, salt temperature ($T_{salt}$) = 900 K, pressure difference ($\Delta p$) = 4e5 Pa, gas outlet diameter ($D_o$) = 0.02, and helium volumetric flow rate ($Q_{He}$) = 0.005 m$^{3}$/s. These design features are well sufficient.

    On the other hand, it seems the separator design is not critical as its efficiency is always above 95\%. Even if we change the separator design parameters by half, the change in the efficiency seems to remain very limited with a few percent change. Therefore, these results indicate that the main difficulty in designing sparging system comes from nothing but the sparger component. It has to be designed according to the removal efficiency imposed by the reactor core. Note that one of the important design parameters which are affecting both sparger and separator is bubble diameter as pointed out in Milestone 1.2.

    The base design which provides an $\varepsilon$$_{Xe}$ of about 40\% may be a good starting point in designing a sparging system. This design seems sufficient in particular for the reactor cores which need less than 40\% gas removal efficiency to compensate the reactivity loss due to Xe poisoning. On the other hand, we need a more adaptable sparger design for a prompt response to a need for a higher gas removal efficiency like $\varepsilon$$_{Xe}$ = 0.769 as indicated in Figure \ref{fig:triple_keff}. In this manner, we are recommending a few sparger designs (for the worst-case scenarios) below: one for $\varepsilon$$_{Xe}$ = 53.6 and one for $\varepsilon$$_{Xe}$ = 76.9\%.

    To reach at the goal of $\varepsilon$$_{Xe}$ = 53.6 implied by Figure \ref{fig:double_keff} for the double load follow transient, it would be sufficient to increase only sparger pipe length and diameter by 30\% in the base design, without touching the gas and salt flow rates. That is, the sparger design should have a pipe length (L) = 13 m, pipe diameter ($d$) = 0.13 m, bubble diameter ($d_b$) = 1 mm, salt volumetric flow rate (Q$_{salt}$) = 0.1 m$^{3}$/s, salt temperature ($T_{salt}$) = 900 K and helium volumetric flow rate (Q$_{He}$) = 0.005 m$^{3}$/s.

    In the case of reaching at the goal of $\varepsilon$$_{Xe}$ = 76.9\% implied by Figure \ref{fig:triple_keff} for the triple load follow transient, it would not be as easy as to go the goal of $\varepsilon$$_{Xe}$ = 53.6\%. This is because there is no way to go to 76.9\% even if we increase the sparger pipe length and diameter by 50\%. Therefore we need additional parameters to adjust. Changing the gas to salt flow rate ratio seems the only way as the other parameters have a very limited impact on the efficiency to increase. In this manner, increasing the gas flow rate and sparger pipe length and diameter by 50\% in the base design yields the required efficiency. That is, the sparger design should have a pipe length (L) = 15 m, pipe diameter ($d$) = 0.15 m, bubble diameter ($d_b$) = 1 mm, salt volumetric flow rate (Q$_{salt}$) = 0.1 m$^{3}$/s, salt temperature ($T_{salt}$) = 900 K and helium volumetric flow rate (Q$_{He}$) = 0.0075 m$^{3}$/s.

    It appears from the results that the best option is to regulate the gas flow rate and/or salt flow rate during operation, or much better is to balance the ratio (by less than 5\%) of gas flow rate to salt flow rate, which is delimited by thermal-hydraulic effects. Another good option is to make the sparger pipe length longer as efficiency reacts equally to change in length, like 10\% for 10\%. It also has no significant effect on the separator.

    To sum up, the main findings in this milestone are: (i) TAP can do load-following without gas removal and safety parameters remained almost constant, (ii) MSBR needs gas removal for the load-following but very high separation efficiency is unnecessary, (iii) We need an adaptable sparging design so that a sufficient gas removal is achievable for all kinds of load-follow scenarios, and (iv) design parameters are very limited by the flow dynamics and core requirements.
